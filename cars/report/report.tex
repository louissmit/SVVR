\documentclass{article}

\usepackage[margin=1in]{geometry}
\usepackage{amsmath,amssymb}
\usepackage{graphicx}

\title{Representing High-Dimensional Datasets}

\author{Luka Stout \\ 10616713  \and Louis Smit \\ ???}

\begin{document}

\maketitle

\section{Introduction}
In Scientific Visualization one of the biggest hurdles one has to overcome is how to correctly visually represent a high-dimensional dataset. Conventional representation methods, such as line graphs, bar charts and pie charts are commonly used to represent 2- or 3D 
datasets, sets with few variables. To represent high-dimensional datasets one has to use other methods. In this paper we present the way we chose to represent our 7D-dataset and explain our choices.

\section{Background}
\textit{Necessary? Probably just Bertin's work or some other landmark papers if possible (Max 2 pages so maybe unnecessary).}

\section{Data}
\textit{Talk a bit about our dataset (Maybe only info that Robert gave us).} The data that we used was a filtered version of the \textit{Common/Old/Convetional/??} dataset first used in \textit{REF}.
% 2. Sources:
%    (a) Origin:  This dataset was taken from the StatLib library which is
%                 maintained at Carnegie Mellon University. The dataset was 
%                 used in the 1983 American Statistical Association Exposition.
%    (c) Date: July 7, 1993

\section{Visual attributes}
\textit{Use Bertin's visual attributes to explain choices.}

\section{Visualization}
\textit{Include figure and explain what it shows. Also explain how we made it}

\section{Conclusion}
In this paper we showed our way to represent the 

\section{Bibliography}

\end{document}
