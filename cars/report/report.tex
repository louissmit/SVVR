\documentclass{article}

\usepackage[margin=1in]{geometry}
\usepackage{amsmath,amssymb}
\usepackage{graphicx}

\title{Representing High-Dimensional Datasets}

\author{Luka Stout \\ 10616713  \and Louis Smit \\ ???}

\begin{document}

\maketitle

\section{Introduction}
In Scientific Visualization one of the biggest hurdles one has to overcome is how to correctly visually represent a high-dimensional dataset. Conventional representation methods, such as line graphs, bar charts and pie charts are commonly used to represent 2- or 3D 
datasets, sets with few variables. To represent high-dimensional datasets one has to use other methods. In this paper we present the way we chose to represent our 7D-dataset and explain our choices.

\section{Background}
\textit{Necessary? Probably just Bertin's work or some other landmark papers if possible (Max 2 pages so maybe unnecessary).}

\section{Data}
The data that we used was an alterend and filtered version of the 'cars' dataset first used in the 1983 American Statistical Association Exposition, which is maintained by the StatLib Library at Carnegie Mellon University\footnote{Found at http://lib.stat.cmu.edu/index.php}. The data was collected by Ernesto Ramos and David Donoho. We no longer use the engine displacement listed in the original dataset. In the data set there are 392 observations about cars produced between 1970 and 1982. Every car is represented by the following data items:
\begin{center}
	\begin{tabular}{| l | l | l |}
	\hline
	\textbf{Column} & \textbf{Description} & \textbf{Detail} \\ \hline
	Model & Model name & \\ \hline
	MPG & Miles per gallon & \\ \hline
	Cylindes & Number of cylinders & 3,4,5,6 or 8 cylinders \\ \hline
	Horsepower & Horsepower &  \\ \hline
	Weight & Vehicle weight in lbs. & \\ \hline
	Year & Model year & Modulo 100. Ranging from 70 to 83\\ \hline
	Origin & Country of original & US = American \\&& Europe = European \\ &&Japan = Japanese \\ \hline
	\end{tabular}
\end{center}

\section{Visual attributes}
\textit{Use Bertin's visual attributes to explain choices.}

\section{Visualization}
\textit{Include figure and explain what it shows. Also explain how we made it}

\section{Conclusion}
In this paper we showed our way to represent the 

\section{References}

\end{document}
