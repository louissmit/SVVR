\documentclass{article}
\usepackage[utf8]{inputenc}
\usepackage[margin=1in]{geometry}
\usepackage{amsmath,amssymb}
\usepackage{graphicx}
\usepackage{capt-of}

\title{Representing High-Dimensional Datasets}

\author{Luka Stout \\ 10616713  \and Louis Smit \\ ???}

\begin{document}

\maketitle

\section{Introduction}
In Scientific Visualization one of the biggest hurdles one has to overcome is how to correctly visually represent a high-dimensional dataset. Conventional representation methods, such as line graphs, bar charts and pie charts are commonly used to represent 2- or 3D 
datasets, sets with few variables. To represent high-dimensional datasets one has to use other methods. In this paper we present the way we chose to represent our 7D-dataset and explain our choices.

\section{Background}
In 1974 Jaques Bertin wrote his now famous paper \emph{Sémiologie Graphique}\cite{Bertin74}. In this paper Bertin described some basic units, including points, lines, and areas, as the basis for multiple methods through which these units can be modified, including position, size, shape and color. These modifications are called visual attributes. Each of these attributes can have certain characteristics. He defined seven of these visual attributes. They consist of:
\begin{center}
	\begin{tabular}{| l | l |}
	\hline
	\textbf{Attribute} & \textbf{Description} \\ \hline
	Position & Changes in the x,y location \\ \hline
	Size & Changes in length or area \\ \hline
	Shape & Changes in shape \\ \hline
	Value & Changes from light to dark \\ \hline
	Colour & Changes in hue \\ \hline
	Orientation & Changes in alignment \\ \hline
	Texture & Variation in texture \\ \hline
	\end{tabular}
	\captionof{table}{Bertin's original Visual Attributes}
\end{center}

These Visual Attributes are used in Scientific Visualization to make visualizations that are unambigous and don't use the same visual attributes to represent different data attributes or represent the same data attributes using multiple visual attributes. Later on this list was expanded and was provided with different sorting for their accuracy, based on the task by Jock D. Mackinlay\cite{Mackinlay86}.

\section{Data}
The data that we used was an altered and filtered version of the 'cars' dataset first used in the 1983 American Statistical Association Exposition, which is maintained by the StatLib Library at Carnegie Mellon University\footnote{Found at http://lib.stat.cmu.edu/index.php}. The data was collected by Ernesto Ramos and David Donoho. We no longer use the engine displacement listed in the original dataset. In the data set there are 392 observations about cars produced between 1970 and 1982. Every car is represented by the following data:
\begin{center}
	\begin{table}
	\begin{tabular}{| l | l | l | l |}
	\hline
	\textbf{Column} & \textbf{Description} & \textbf{Detail} & \textbf{Nominal, Ordinal or Quantitative} \\ \hline
	Model & Model name & & Nominal \\ \hline
	MPG & Miles per gallon & & Quantitative \\ \hline
	Cylinders & Number of cylinders & 3,4,5,6 or 8 cylinders & Ordinal\\ \hline
	Horsepower & Horsepower & & Quantitative \\ \hline
	Weight & Vehicle weight in lbs. & & Quantitative\\ \hline
	Year & Model year & Modulo 100. Ranging from 70 to 83 & Ordinal\\ \hline
	Origin & Country of original & US = American & Nominal \\&& Europe = European & \\ &&Japan = Japanese & \\ \hline
	\end{tabular}
	\label{table:data_desc}
	\end{table}
	\captionof{table}{Dataset description.}
\end{center}

\section{Visual attributes}
To start representing this dataset in one visualization we looked at every variable and whether it was a Nominal, Ordinal or Quantative variable. This is done so we can accurately represent every variable by using only one visual attribute and only use every visual attribute only once. In Table \ref{table:data_desc} it can be seen that there are multiple quantitative variables, however three quantitative variables can be represented easily in a 2D-visualization by using position on the x-axis, position of the y-axis and scale of the points that represent every dataitem. So Miles Per Gallon (MPG) is being represented by using the position of the x-axis, horsepower on the y-axis and every point is being scaled using the weight of the vehicle. The ordinal variables are being represented by using shape for the amount of cylinders, where the saturation of the color of every datapoint is being used for the build year of the car. The last variable, country of origin is being depicted by the hue of the color of every datapoint. This last choice was made early on in the process of making this visualization to show the difference between cars from different continents.

\section{Visualization}
\textit{Include figure and explain what it shows. Also explain how we made it}

\section{Conclusion}
In this paper we showed our way to represent the 

\bibliographystyle{plain}
\begin{thebibliography}{99}
\bibitem{Bertin74}
Bertin, J.: \emph{Sémiologie Graphique}. Paris: Editions Gauthier-Villars. Deutsche Übersetzung von Jensch, G.; Schade, D.; Scharfe, W.: Graphische Semiologie. Diagramme – Netze - Karten. Berlin: Walter de Gruyter, 1974.
\bibitem{Mackinlay86}
Mackinlay, J. D.: \emph{Automating the design of graphical presentations of relational information.}1986. ACM Trans. Graph. 5, 2 (April 1986), 110-141
\end{thebibliography}

\end{document}
